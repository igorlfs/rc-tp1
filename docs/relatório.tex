\documentclass[12pt]{article}

\usepackage{fontspec}           % Pediram para usar Arial 12
\setmainfont{Arial}
\usepackage[T1]{fontenc}        % Codificação para português (ao copiar e colar do documento, acentos são colados corretamente)
\usepackage[portuguese]{babel}  % Português (alguns pacotes usam termos da linguagem, por ex 'Referências')

\author{Igor Lacerda}
\title{Trabalho Prático I}
\date{Redes de Computadores}

\begin{document}

\maketitle

O objetivo do trabalho foi aprender sobre a interação servidor-cliente usando sockets, na linguagem C, por meio da implementação de um clone do clássico Campo Minado. Esse objetivo foi atingido com sucesso.\footnote{A formatação desse trabalho não está tão elegante quanto deveria estar, porque eu não tenho as variantes (itálico, negrito) da fonte \textit{Arial} no meu computador. Para uma versão um pouco mais bonita, considere trocar a fonte para o padrão do \LaTeX\ ou recompilar o documento se tiver as fontes.}

\section*{Desafios, Dificuldades e Imprevistos}

A maior dificuldade, foi, indiscutivelmente, começar. Como alguém que não presta atenção nas aulas, eu não sabia quais funções deveria usar para realizar a conexão e afins. Felizmente existe uma quantidade abundante de recursos sobre esse tema na \textit{internet}. Agora é completamente natural para mim, realizar uma conexão criando uma \textit{socket}, definindo os parâmetros (protocolo, endereço, porta), fazendo a atribuição dos parâmetros para a \textit{socket} e assim por diante. A princípio pode parecer um processo desnecessariamente complicado, mas ao final, é compreensível ser do jeito que é.

Outra dificuldade foi o uso do C, que é uma linguagem \emph{intankável}, em que toda e qualquer coisa é mais complicada do que deveria ser (isso é sabido; é exatamente esse o motivo pelo qual pedem para que a gente faça em C). Eu muito preferiria ter feito esse trabalho em Rust. Mas até que não foram tantos impedimentos nesse projeto especificamente: alguns \textit{segfaults} aqui e ali, uma saudade eterna de um tipo \textit{string}...

Outra complicação que, ao meu ver, pareceu desnecessária, foi o suporte a tanto IPv4 como IPv6, que exigiu certa engenhosidade para lidar com os tipos \texttt{sockaddr\_in} e \texttt{sockaddr\_in6}, bem como o uso da função \texttt{inet\_pton} no cliente, para saber a versão do protocolo. Não achei minhas soluções particularmente elegantes no que diz respeito a isso.

Certamente os maiores aborrecimentos durante o projeto foram: o sistema operacional demorando para liberar a porta e eu esquecendo de reiniciar o servidor quando estava testando o cliente. Isso acabou atrasando um pouco as coisas (em parte porque é bem fácil eu ficar distraído) e introduziu certa confusão, porque às vezes parecia que os programas tinham parado de funcionar (e às vezes realmente tinham).

Em relação ao desenvolvimento (a lógica) do jogo em si, não houve grandes dificuldades, como é de se esperar. Mas alguma quebra de cabeça foi necessária para resolver alguns \textit{bugs} chatinhos (nenhum caso particular que mereça um comentário). No aspecto de engenharia de \textit{software}, inicialmente os programas consistiam apenas na função principal, mas tentei extrair alguns blocos em funções e no geral deixar as coisas menos bagunçadas, mas não foi um esforço tão exaustivo.

Tendo concluído o trabalho, diria que ele não foi muito difícil, apesar de ser mais complicado do que deveria. Achei bacana ver como o conteúdo de redes é usado na prática e o tema escolhido, com um joguinho, é uma ótima ideia.

\end{document}
